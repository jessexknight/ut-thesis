%%%%%%%%%%%%%%%%%%%%%%%%%%%%%%%%%%%%%%%%%%%%%%%%%%%%%%%%%%%%%%%%%%%%%%
%%  
%%  UT-THESIS.TEX
%%  Copyright (c) 1998-2000 by Francois Pitt
%%  Last Update: 2000 February 20
%%  
%%  Skeleton LaTeX2e file for the preparation of theses at UofT;
%%  conforms to the School of Graduate Studies' guidelines of 07/97.
%%  To be used in conjunction with class file `ut-thesis.cls', whose
%%  features it illustrates.
%%  
%%%%%%%%%%%%%%%%%%%%%%%%%%%%%%%%%%%%%%%%%%%%%%%%%%%%%%%%%%%%%%%%%%%%%%
%%  
%%  This file is distributed in the hope that it will be useful but
%%  without any warranty (without even the implied warranty of
%%  fitness for a particular purpose).  For a description of this
%%  file's purpose, and instructions on its use, see below.
%%  
%%  Feel free to copy and redistribute this file, as long as this
%%  copyright notice remains intact and this file is distributed
%%  along with the companion file `ut-thesis.cls'.
%%  
%%  (Thanks to Robert Bernecky for his suggestions on improving the
%%  usefulness and readability of this file.)
%%  
%%  Send all bugs, questions, comments, suggestions, etc. to the
%%  author, at <fpitt_AT_cs_DOT_utoronto_DOT_ca>.
%%  
%%%%%%%%%%%%%%%%%%%%%%%%%%%%%%%%%%%%%%%%%%%%%%%%%%%%%%%%%%%%%%%%%%%%%%
%%  
%%  SUMMARY OF FEATURES:
%%  
%%  To comment out parts of a file, use the macro \ignore{...}
%%  around the entire block of text you want to ignore.
%%  
%%  To explicitly set the pagestyle of any inserted blank page when
%%  \cleardoublepage occurs, use one of \clearemptydoublepage or
%%  \clearplaindoublepage instead.
%%  
%%  For single-spaced quotes or quotations, use the `longquote' and
%%  `longquotation' environments.  For single-spaced, 1 1/2-spaced,
%%  or double-spaced paragraphs, use one of the environments
%%  `singlespaced', `oneandahalfspaced', or `doublespaced'.  More
%%  generally, for paragraphs with a line spacing of `n', use
%%  `\begin{newspacing}{n}...\end{newspacing}'.
%%  
%%  All other environments, commands, and options provided by the
%%  `ut-thesis' class will be described below, at the point where
%%  they should appear in the document.
%%  
%%  See the companion file `ut-thesis.cls' for more details.
%%  
%%%%%%%%%%%%%%%%%%%%%%%%%%%%%%%%%%%%%%%%%%%%%%%%%%%%%%%%%%%%%%%%%%%%%%


%%%%%%%%%%%%         PREAMBLE         %%%%%%%%%%%%

%% Default settings format a final copy (12pt font, single-sided,
%% double-spaced, normal margins, single-spaced notes).  For a rough
%% copy (10pt font, double-sided, double-spaced, normal margins, with
%% the word "DRAFT" printed at each corner of every page), use the
%% `draft' option.  The default line spacing can be changed with one
%% of the following options: `singlespaced', `oneandahalfspaced', or
%% `doublespaced'.  The notes are always single-spaced by default, but
%% can be made to have the same spacing as the rest of the document by
%% using the option `spacednotes'.  The size of the margins can be
%% changed with one of the following options: `narrowmargins' (1 1/4"
%% left, 3/4" others), `normalmargins' (1 1/4" left, 1" others),
%% `widemargins' (1 1/4" all), `extrawidemargins' (1 1/2" all).  Any
%% other standard option for the `report' document class can be used
%% to override the default or draft settings.

%% ***   Add any desired options.   ***
\documentclass{ut-thesis}

%% ***   Add \usepackage declarations here.   ***


%% The line spacing of the document should be specified using one of
%% the document options given above, but if you need a line spacing
%% that is not provided by the options, you can override the default
%% line spacing for the entire document with the command
%%   `\linespacing{...}'.
%% Note that in order to get the correct appearance, the argument to
%% `\linespacing' must be equal to 1/3 + 2/3 times the desired line
%% spacing (for example, single-spaced = \linespacing{1},
%%                        1 1/2-spaced = \linespacing{1.33}, and
%%                       double-spaced = \linespacing{1.66}).

%% ***   Uncomment and fill in a value, if needed.    ***
%% ***   REMEMBER: You should NOT need to use this.  Use one of   ***
%% ***   the document class options mentionned above instead.     ***
%\linespacing{}

%%%%%%%%%%%%%%%%%%%%%%%%%%%%%%%%%%%%%%%%%%%%%%%%%%%%%%%%%%%%%%%%%%%%%%
%%                                                                  %%
%%                  ***   I M P O R T A N T   ***                   %%
%%                                                                  %%
%%  Fill in the following fields with the required information:     %%
%%   - \degree{...}       name of the degree obtained               %%
%%   - \department{...}   name of the graduate department           %%
%%   - \gradyear{...}     year of graduation                        %%
%%   - \author{...}       name of the author                        %%
%%   - \title{...}        title of the thesis                       %%
%%%%%%%%%%%%%%%%%%%%%%%%%%%%%%%%%%%%%%%%%%%%%%%%%%%%%%%%%%%%%%%%%%%%%%

%% ***   Change this example to appropriate values.   ***
\degree{Doctor of Philosophy}
\department{Computer Science}
\gradyear{1999}
\author{Fran\c{c}ois Pitt}
\title{UT-Thesis Class File Example}

%% ***   NOTE   ***
%% Put here all other formatting commands that belong in the preamble.
%% In particular, you should put all of your `\newcommand',
%% `\newenvironment', and `\newtheorem's (in other words, all the
%% global definitions that you will need throughout your thesis) in
%% a separate file and use "\input{filename}" to input it here.


%% As another example, to list only down to subsections in the table
%% of contents (-1=part, 0=chapter, 1=section, 2=subsection,
%% 3=subsubsection, 4=paragraph, 5=subparagraph, 6=subsubparagraph).
%
\setcounter{tocdepth}{2}


%%%%%%%%%%%%      MAIN  DOCUMENT      %%%%%%%%%%%%

\begin{document}

%% This sets the page style and numbering for preliminary sections.
\begin{preliminary}

%% This generates the title page from the information given above.
\maketitle

%% There should be NOTHING between the title page and abstract.

%% This generates the abstract page, with the line spacing adjusted
%% according to SGS guidelines.
\begin{abstract}
%% ***   Put your Abstract here.   ***
%% (At most 150 words for M.Sc. or 350 words for Ph.D.)
\end{abstract}

%% Anything placed between the abstract and table of contents will
%% appear on a separate page since the abstract ends with \newpage
%% and the table of contents starts with \clearpage.

%% This generates a "dedication" section, if needed.
%% (uncomment to have it appear in the document)
%\begin{dedication}
%% ***   Put your Dedication here.   ***
%\end{dedication}

%% The `dedication' and `acknowledgements' sections do not create new
%% pages so if you want the two sections to appear on separate pages,
%% you should put an explicit \newpage between them.

%% This generates an "acknowledgements" section, if needed.
%% (uncomment to have it appear in the document)
%\begin{acknowledgements}
%% ***   Put your Acknowledgements here.   ***
%\end{acknowledgements}

%% This generates the Table of Contents (on a separate page).
\tableofcontents

%% This generates the List of Tables (on a separate page), if needed.
%% (uncomment to have it appear in the document)
%\listoftables

%% This generates the List of Figures (on a separate page), if needed.
%% (uncomment to have it appear in the document)
%\listoffigures

%% You can add commands here to generate any other material that
%% belongs in the head matter (for example, an "Index of Symbols").

%% End of the preliminary sections: reset page style and numbering.
\end{preliminary}

%%%%%%%%%%%%%%%%%%%%%%%%%%%%%%%%%%%%%%%%%%%%%%%%%%%%%%%%%%%%%%%%%%%%%%
%%  Put your Chapters here; the easiest way to do this is to keep   %%
%%  each chapter in a separate file and `\include' all the files    %%
%%  right here.  Note that each chapter file should start with the  %%
%%  line "\chapter{ChapterName}".  Note that using `\include'       %%
%%  instead of `\input' makes each chapter start on a new page,     %%
%%  and allows you to format only parts of your thesis at a time    %%
%%  by using `\includeonly'.                                        %%
%%%%%%%%%%%%%%%%%%%%%%%%%%%%%%%%%%%%%%%%%%%%%%%%%%%%%%%%%%%%%%%%%%%%%%

%% ***   Include chapter files here.   ***


%% This adds a line for the Bibliography in the Table of Contents.
\addcontentsline{toc}{chapter}{Bibliography}
%% ***   Set the bibliography style.   ***
%% (change according to your preference)
\bibliographystyle{plain}
%% ***   Set the bibliography file.   ***
%% ("thesis.bib" by default; change if needed)
\bibliography{thesis}

%% ***   NOTE   ***
%% If you don't use bibliography files, comment out the previous line
%% and use \begin{thebibliography}...\end{thebibliography}.  (In that
%% case, you should probably put the bibliography in a separate file
%% and `\include' or `\input' it here).

\end{document}

%%%%%%%%%%%%%%%%%%%%%%%%%%%%%%%%%%%%%%%%%%%%%%%%%%%%%%%%%%%%%%%%%%%%%%
%%  End of UT-THESIS.TEX
%%%%%%%%%%%%%%%%%%%%%%%%%%%%%%%%%%%%%%%%%%%%%%%%%%%%%%%%%%%%%%%%%%%%%%
